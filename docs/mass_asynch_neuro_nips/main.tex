\documentclass{article} % For LaTeX2e
\usepackage{nips15submit_e,times}

\usepackage{hyperref}
\usepackage{url}
\usepackage[toc,page]{appendix}

%citation
\usepackage[backend=bibtex]{biblatex}
\bibliography{openbrain}
%

% tikz and associated macros
\usepackage{tikz}

\usetikzlibrary{decorations.pathmorphing}
\tikzset{snake it/.style={decorate, decoration=snake}}
%

% math
\usepackage{amsthm}
\usepackage{amsmath}
\usepackage{amssymb}
\usepackage{mathabx}
\newcommand{\BlackBox}{\rule{1.5ex}{1.5ex}}  % end of proof
\newtheorem{example}{Example} 
\newtheorem{theorem}{Theorem}
\newtheorem{lemma}[theorem]{Lemma} 
\newtheorem{proposition}[theorem]{Proposition} 
\newtheorem{remark}[theorem]{Remark}
\newtheorem{corollary}[theorem]{Corollary}
\newtheorem{definition}[theorem]{Definition}
\newtheorem{conjecture}[theorem]{Conjecture}
\newtheorem{axiom}[theorem]{Axiom}

\numberwithin{equation}{subsection}
\numberwithin{theorem}{subsection}

\DeclareSymbolFont{cmlargesymbols}{OMX}{cmex}{m}{n}b
\let\sumop\relax
\DeclareMathSymbol{\sumop}{\mathop}{cmlargesymbols}{"50}

%

%% todo[inline] NOTES. To remove for camera ready version.
\usepackage{todonotes}
\usepackage{regexpatch}
%end to notes

\title{OpenBrain: Massively Asynchronous Neurocomputation}


\author{
William H.~Guss \\
Machine Learning at Berkeley\\
2650 Durant Ave, Berkeley CA, 94720 \\
\texttt{wguss@ml.berkeley.edu} \\
\And
Mike Zhong \\
Machine Learning at Berkeley \\
Berkeley CA, 94720 \\
\texttt{lol@gmail.com} \\
\And
Phillip Kuznetsov \\
Machine Learning at Berkeley \\
Berkeley CA, 94720 \\
\texttt{philkuz@ml.berkeley.edu} \\
\And
Jacky Liang \\
Machine Learning at Berkeley \\
Berkeley CA, 94720 \\
\texttt{jackyliang@berkeley.edu} \\
\And
Instert yourself \\
Machine Learning at Berkeley \\
Address \\
\texttt{email} \\
\And
We can add footnotes clarifying \\
Machine Learning at Berkeley \\
Address \\
\texttt{email} \\
\And
equal contributions to the work \\
(if needed)\\
Machine Learning at Berkeley \\
Address \\
\texttt{email} 
}


\nipsfinalcopy % Uncomment for camera-ready version

\begin{document}


\maketitle

\begin{abstract}
	In this paper we introduce a new framework and philosophy for recurrent neurocomputation. By requiring that all neurons act asynchrynously and independently, we introduce a new metric for evaluating the universal intelligence of continuous time agents. We proved representation and universal approximation results in this context which lead to a set of learning rules in the spirt of John Conway's game of life. Finally evaluate this framework against different intelligent agent algorithms by implementing an approximate universal intelligence measure for agents embedded in turing computable environments in Minecraft, BF, and a variety of other reference machines. 
\end{abstract}


\listoftodos
%!TEX root = main.tex
\section{Introduction}

Most standard neural network models depart from the biological neuron in a number of ways. Our asynchronous continuous-time model draws inspiration from physiological and biological principles for neuron dynamics.

In a biological neuron, there are leak Na+ and K+ current channels that constantly push the neuron's voltage towards its equilibrium voltage. We have included an exponential decay term to model this homeostatic tendency towards equilibrium. Likewise, we have also loosely modelled the refractory period, the period immediately after an action period in which a neuron cannot send a signal, instead of having resetting the voltage of a neuron immediately back to zero.

There are many features of a biological neuron that we have chosen not to implement (most notably the dynamics of a neuron's voltage during a action potential, by implementing a Hodgkins-Huxley model (see wikipedia)) in that there is a trade-off between having a model that is more biological realisticity and one that is mathematically and computationally simple. There are projects like BlueBrain that aim for the most biologically realistic model, and a plethora of projects (deep learning, etc.) that are vaguely biologically inspired but have been massively implemented and successful at a small class of tasks. Our hope is by striving for a balance between the two ideals by implementing a system that is both biologically inspired and computationally simple, we would be able to make a massive implementation that succeeds in a wide variety of tasks that require both creativity and intelligence.

%\emph{Note:} 
%We need to motivate substantially the proposal of a drastically different framework of neurocomputation. This motivation should be given according to the latest neuroscience, and a general problem in the field.

%\todo[inline]{Argue against 
%the current state of ML's approach to the problem of AGI
%by surveying field.}
%\subsection{Intelligence as an Emergence Phenomenon}
%\todo[inline]{Establish Conwaynian philosophy on emergent neurocomputation.}
%\todo[inline]{Motivate asynchrynous neurocomputation.}

%!TEX root = main.tex
\section{Asynchronous Neurocomputation}
\emph{Note:} This section will essentially lay out our model minus learning rules. This means we must give theoretical, biological justifications for the algorithm.

%!TEX root = ../main.tex
\subsection{The Core Framework}

%%\todo[inline]{This section needs an actuall discussion. For now here are the working definitions of Openbrain.}
We begin by giving basic abstractions on neurons.

\begin{definition}\label{neuron}
	A \textbf{neuron} $n \in N$ is defined by
	\begin{itemize}
		\item a voltage $V_n(t)$
		\item a decay time $\tau_n$
		\item a refactory period $\rho_n$
		\item a voltaic threshold $\theta_n$
	\end{itemize}
\end{definition}
\begin{definition}\label{connection}
	A \textbf{connection} $c \in C$ is a tuple $(n_i, n_j, w_{ij}) \in N \times N \times \mathbb{R}$
	where $n_i$ is  the \textbf{anterior neuron}, $n_j$ is the \textbf{posterior neuron}, and $w_ij$
	is the standard synaptic weight.
\end{definition}

%\todo[inline]{Make connection diagram.}


For a neuron $n$, we denote the set of anterior neurons $A_n$ and the dendritic connections, $C_{n}^a$. In the same light we will use the notations $P_n$ and $C_n^p$ to denote the sets of posterior neurons and posterior connections for $n$ respectively.

Because it is impossible to implement time as a continuous parameter on any computer, we introduce the time-step, $\Delta t$, as a numerical simulation step-size in time. The conversion between discrete algorithm iterations, with $k = 0, 1, 2, ...$ and continuous time $t$ is $V(t_{k}) = V(k\Delta t) = V[k]$. As computational power permits, a smaller $\Delta t$ gives a more accurate simulation.

\begin{definition}\label{decay}
	We say that a neuron $n$ experiences \textbf{voltage decay} so that for all $k$,
	\begin{equation}
		V_n[k+1] \leftarrow V_n[k]e^{-\Delta t/\tau}. \label{eq:decay}
	\end{equation}
	so that unless it obtains voltage from anterior neurons' firing, its voltage will decay exponentially to 0.
\end{definition}


\begin{definition}\label{fire}
	A neuron $n$ is said to \textbf{fire} if it is not in its refractory period and $V_n(t_{k}) = V_n[k] > \theta_n$. Then for all $m \in P_n$,
	\begin{equation}
		V_m[k+1] += w_{nm} \sigma(V_n[k]); \label{eq:fire}
	\end{equation}
	that is, voltage is propagated to the posterior neurons. 	Immediately after neuron $n$ fires, it enters  a \textbf{refractory period} until time $t_{k} + \rho_n$, or iteration $k + \frac{\rho_n}{\Delta t}$.
\end{definition}


Combining \eqref{eq:fire} and \eqref{eq:decay} we get that for a neuron $m$ at time $k+1$
\begin{equation}
	V_m[k+1] = V_m[k]e^{-\Delta  t/\tau} + \sumop_{\substack{n \in A'_m}} w_{nm} \sigma(V_n[k])
\end{equation}
such that $A'_m$ is the set of anterior neurons which fired at time $k.$

%!TEX root = main.tex
\subsection{Continuous Time Universal Intelligence Measure}
%!TEX root = main.tex
\subsection{Universal Approximation}
\subsection{(optional) Multiprocess Turing Completeness}
If we decide to go down this route, we'll add it's own \TeX file.
%!TEX root = main.tex
\section{Asynchronous Signal Error Backpropagation}

In this section we introduce an asynchronous signal error backpropagation method,
as a general framework for updating parameters using the policy gradient. In traditional
supervised learning, neural networks propagate an error gradient to almost every neuron on every layer.
This method is adventageous since it it is easily optimized using GPUs and tensor multiplication, but measures must be 
taken to introduce sparsity and prevent overfitting such as dropout and ReLU activations. 

Our framework immediately invokes sparsity of both neuronal connections and neuronal firings and therefore would perform suboptimally 
by such methods. Fortunately our framework can take advantage of chain rule as a graph transition process. Consider the following situation.

\begin{definition}
	Select subsets of neurons $N_I, N_O$ to denote the i\textbf{nput and output neuron}s for an openbrain $O.$ Then given a dataset $D$ of datapairs $(x,y)$ where $x \in \mathbb{R}^{|N_I|}, y \in \mathbb{R}^{|N_O|}$, we define the \textbf{error or loss} of $O$ on $D$ as 
	\begin{equation}
		\begin{aligned}
			V(n,t) &\leftarrow x, &n \in N_I \\
			E(t+t_e) &= \frac{1}{2} \|V(m,t+t_e) - y\|^2, &m \in N_O
		\end{aligned}
	\end{equation}
	where $(x,y) \in D$ and $[t,t+t_e]$ is the \textbf{evaluation interval}.
\end{definition}

Learning is then defined as the optimization of the error function over the evaluation interval with respect to the connection parameters which lead to the final behaviour. One might optimize learning so that the most optimal behaviour minimizes the evaluation interval by letting $t_e \to 0$ as $t \to \infty,$ but we hold $t_e$ constant in the initial formulation.


Suppose we have a neuron $n$ as in \todo{Insert neuron figure here} that has just been activated at $t = t_0.$ This leads to a series of activations tagged with the activation of $n$ at $t_0$ 


\section{Experimentation [Omitted]}
\subsection{Implementation [Omitted]}
\subsection{Results [Omitted]}
\section{Conclusion [Omitted]}
\subsection{Future Work [Omitted]}

\printbibliography

\begin{appendices}
\section{Universal Intelligence Definitions}
\end{appendices}

\end{document}