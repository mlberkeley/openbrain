\documentclass[final]{beamer}

\usepackage[scale=1.24]{beamerposter} % Use the beamerposter package for laying out the poster

\usetheme{confposter} % Use the confposter theme supplied with this template

\setbeamercolor{block title}{fg=black,bg=white} % Colors of the block titles
\setbeamercolor{block body}{fg=black,bg=white} % Colors of the body of blocks
\setbeamercolor{block alerted title}{fg=white,bg=dblue!70} % Colors of the highlighted block titles
\setbeamercolor{block alerted body}{fg=black,bg=dblue!10} % Colors of the body of highlighted blocks
% Many more colors are available for use in beamerthemeconfposter.sty

%-----------------------------------------------------------
% Define the column widths and overall poster size
% To set effective sepwid, onecolwid and twocolwid values, first choose how many columns you want and how much separation you want between columns
% In this template, the separation width chosen is 0.024 of the paper width and a 4-column layout
% onecolwid should therefore be (1-(# of columns+1)*sepwid)/# of columns e.g. (1-(4+1)*0.024)/4 = 0.22
% Set twocolwid to be (2*onecolwid)+sepwid = 0.464
% Set threecolwid to be (3*onecolwid)+2*sepwid = 0.708

\newlength{\sepwid}
\newlength{\onecolwid}
\newlength{\twocolwid}
\newlength{\threecolwid}
\setlength{\paperwidth}{48in} % A0 width: 46.8in
\setlength{\paperheight}{36in} % A0 height: 33.1in
\setlength{\sepwid}{0.024\paperwidth} % Separation width (white space) between columns
\setlength{\onecolwid}{0.22\paperwidth} % Width of one column
\setlength{\twocolwid}{0.464\paperwidth} % Width of two columns
\setlength{\threecolwid}{0.708\paperwidth} % Width of three columns
\setlength{\topmargin}{-0.5in} % Reduce the top margin size


\setbeamertemplate{headline}{
 \leavevmode
  \begin{columns}
  \begin{column}{.04\linewidth}
  \end{column}
     \begin{column}{.15\linewidth}
\hfill \includegraphics[width=1\linewidth]{logo.png} 
   \end{column}
   \begin{column}{.73\linewidth}
    \vskip1cm
    \centering
    \usebeamercolor{title in headline}{\color{jblue}\Huge{\textbf{\inserttitle}}\\[0.5ex]}
    \usebeamercolor{author in headline}{\color{fg}\Large{\insertauthor}\\[1ex]}
    \usebeamercolor{institute in headline}{\color{fg}\large{\insertinstitute}\\[1ex]}
    \vskip1cm
   \end{column}
   \begin{column}{.17\linewidth}
   \end{column}
   \vspace{1cm}
  \end{columns}
 \vspace{0.5in}
 \hspace{0.5in}\begin{beamercolorbox}[wd=47in,colsep=0.15cm]{cboxb}\end{beamercolorbox}
 \vspace{0.1in}
}

%-----------------------------------------------------------

\usepackage{graphicx}  % Required for including images

\usepackage{booktabs} % Top and bottom rules for tables

%----------------------------------------------------------------------------------------
%	TITLE SECTION 
%----------------------------------------------------------------------------------------

\title{Open Brain: Massively Asynchronous Neurocomputation} % Poster title

\author{William Guss, Mike Zhong, Phillip Kuznetsov, Max Johansen, Jacky Liang, Utharsh Singhal, Ananth Kumar} % Author(s)

\institute{Machine Learning at Berkeley} % Institution(s)

%----------------------------------------------------------------------------------------

\begin{document}

\addtobeamertemplate{block end}{}{\vspace*{2ex}} % White space under blocks
\addtobeamertemplate{block alerted end}{}{\vspace*{2ex}} % White space under highlighted (alert) blocks

\setlength{\belowcaptionskip}{2ex} % White space under figures
\setlength\belowdisplayshortskip{2ex} % White space under equations

\begin{frame}[t] % The whole poster is enclosed in one beamer frame

\begin{columns}[t] % The whole poster consists of three major columns, the second of which is split into two columns twice - the [t] option aligns each column's content to the top

\begin{column}{\sepwid}\end{column} % Empty spacer column

\begin{column}{\onecolwid} % The first column

%----------------------------------------------------------------------------------------
%	OBJECTIVES
%----------------------------------------------------------------------------------------

\begin{alertblock}{Abstract}

In this project we introduce a new framework and philosophy for recurrent neurocomputation. By requiring that all neurons act asynchrynously and independently,
we introduce a new metric for evaluating the universal intelligence of continuous
time agents. We are experimenting with a set of simple learning rules in the spirt of John Conway's game of life. Finally we evaluate this framework against different intelligent agent algo-
rithms by implementing an approximate universal intelligence measure for agents
embedded in turing computable environments in Minecraft, BF, and a variety of
other reference machines.
\end{alertblock}

%----------------------------------------------------------------------------------------
%	INTRODUCTION
%----------------------------------------------------------------------------------------

\begin{block}{Introduction}
Most standard neuron network models depart from the biological neuron in a number of ways. Our asynchronous continuous-time model draws inspiration from physiological and biological principles for neuron dynamics. By modeling dynamics of the exponential decay of neuron voltage towards equilibrium, refractory periods, and numerous synaptogenesis phenomenons we aim to build a neural architecture that is both biologically inspired and computationally simple. 
\end{block}

%------------------------------------------------


 \begin{block}{Open Brain Definitions}
 A \textbf{neuron} $n \in N$ is defined by:
 \begin{itemize}
 \item voltage $V_n(t)$
 \item decay time $\tau_n$
 \item refractory period $\rho_n$
 \item voltaic threshold $\theta_n$
 \end{itemize}
  A \textbf{connection} $c \in C$ is a tuple $(n_i, n_j, w_{ij}) \in N \times N \times \mathbb{R}$
  where $n_i$ is  the \textbf{anterior neuron}, $n_j$ is the \textbf{posterior neuron}, and $w_ij$
  is the standard synaptic weight.
\end{block}
%----------------------------------------------------------------------------------------

\end{column} % End of the first column

\begin{column}{\sepwid}\end{column} % Empty spacer column

\begin{column}{\twocolwid} % Begin a column which is two columns wide (column 2)

\begin{columns}[t,totalwidth=\twocolwid] % Split up the two columns wide column

\begin{column}{\onecolwid}\vspace{-.6in} % The first column within column 2 (column 2.1)

%----------------------------------------------------------------------------------------
%	MATERIALS
%----------------------------------------------------------------------------------------


\begin{block}{Neuronal Dynamics}
  A neuron $n$ is said to \textbf{fire} if it is not in its refractory period and $V_n(t_{k}) = V_n[k] > \theta_n$. Then for all $m \in P_n$,
  \begin{equation*}
    V_m[k+1] {\ +_=\ } w_{nm} \sigma(V_n[k]); \label{eq:fire}
  \end{equation*}
  that is, voltage is propagated to the posterior neurons.
  Immediately after neuron $n$ fires, it enters  a \textbf{refractory period} until time $t_{k} + \rho_n$, or iteration $k + \frac{\rho_n}{\Delta t}$. 
  	We say that a neuron $n$ experiences \textbf{voltage decay} so that for all $k$,
	\begin{equation}
		V_n[k+1] \leftarrow V_n[k]e^{-\Delta t/\tau}. \label{eq:decay}
	\end{equation}
	so that unless it obtains voltage from anterior neurons' firing, its voltage will decay exponentially to 0.

\end{block}

%----------------------------------------------------------------------------------------

\end{column} % End of column 2.1

\begin{column}{\onecolwid}\vspace{-.6in} % The second column within column 2 (column 2.2)

%----------------------------------------------------------------------------------------
%	METHODS
%----------------------------------------------------------------------------------------

\begin{block}{Learning}

\end{block}

%----------------------------------------------------------------------------------------

\end{column} % End of column 2.2

\end{columns} % End of the split of column 2 - any content after this will now take up 2 columns width

%----------------------------------------------------------------------------------------
%	IMPORTANT RESULT
%----------------------------------------------------------------------------------------

\begin{alertblock}{Important Result}
\textbf{Theorem.}

\end{alertblock} 

%----------------------------------------------------------------------------------------

\begin{columns}[t,totalwidth=\twocolwid] % Split up the two columns wide column again

\begin{column}{\onecolwid} % The first column within column 2 (column 2.1)

%----------------------------------------------------------------------------------------
%	UNIVERSAL INTELLIGENCE
%----------------------------------------------------------------------------------------

\begin{block}{Universal Intelligence Measure in Continuous Time}
A well established\cite{}  machine intelligence measure is the Universal Intelligence Measure proposed by Legg and Hutter \cite{}, whose work developed a consise definition of the intelligence of an \emph{agent} in an \emph{environment}:
	If $\pi$ is an agent then we say that the \textbf{universal intelligence} of $\pi$ is
	\begin{equation} 
		\Upsilon(\pi) = \sum_{\mu \in E}2^{-K(\mu)}V_\mu^\pi
	\end{equation}
	where $V_\mu^\pi$ is the expected reward of the agent in $\mu$,
	\begin{equation}
		V_\mu^\pi = \mathbb{E}\left(\sum_{i=1}^\infty r_i\right) \leq 1.
	\end{equation}
To make a continuous time intelligence measure which is compatible with agents who act instantaneously within an environment, we ??

\end{block}

%----------------------------------------------------------------------------------------

\end{column} % End of column 2.1

\begin{column}{\onecolwid} % The second column within column 2 (column 2.2)

%----------------------------------------------------------------------------------------
%	RESULTS
%----------------------------------------------------------------------------------------

\begin{block}{Project Status}

\begin{figure}
\includegraphics[width=0.8\linewidth]{placeholder.jpg}
\caption{Figure caption}
\end{figure}

Nunc tempus venenatis facilisis. Curabitur suscipit consequat eros non porttitor. Sed a massa dolor, id ornare enim:

\end{block}

%----------------------------------------------------------------------------------------

\end{column} % End of column 2.2

\end{columns} % End of the split of column 2

\end{column}

\begin{column}{\sepwid}\end{column} % Empty spacer column


\begin{column}{\onecolwid} % The third column

%----------------------------------------------------------------------------------------
%	CONCLUSION
%----------------------------------------------------------------------------------------

\begin{block}{Conclusion and Goals}

\end{block}

%----------------------------------------------------------------------------------------
%	ADDITIONAL INFORMATION
%----------------------------------------------------------------------------------------

\begin{block}{Additional Information}


\end{block}

%----------------------------------------------------------------------------------------
%	REFERENCES
%----------------------------------------------------------------------------------------

\begin{block}{References}

\nocite{*} % Insert publications even if they are not cited in the poster
\small{\bibliographystyle{unsrt}
\bibliography{sample}\vspace{0.75in}}

\end{block}

%----------------------------------------------------------------------------------------
%	ACKNOWLEDGEMENTS
%----------------------------------------------------------------------------------------

\setbeamercolor{block title}{fg=red,bg=white} % Change the block title color

\begin{block}{Acknowledgements}

\small{\rmfamily{Nam mollis tristique neque eu luctus. Suspendisse rutrum congue nisi sed convallis. Aenean id neque dolor. Pellentesque habitant morbi tristique senectus et netus et malesuada fames ac turpis egestas.}} \\

\end{block}

%----------------------------------------------------------------------------------------
%	CONTACT INFORMATION
%----------------------------------------------------------------------------------------

\setbeamercolor{block alerted title}{fg=black,bg=norange} % Change the alert block title colors
\setbeamercolor{block alerted body}{fg=black,bg=white} % Change the alert block body colors

\begin{alertblock}{Contact Information}

\begin{itemize}
\item Web: \href{http://www.university.edu/smithlab}{http://www.university.edu/smithlab}
\item Email: \href{mailto:john@smith.com}{john@smith.com}
\item Phone: +1 (000) 111 1111
\end{itemize}

\end{alertblock}



%----------------------------------------------------------------------------------------

\end{column} % Endz of the third column

\end{columns} % End of all the columns in the poster

\end{frame} % End of the enclosing frame

\end{document}
