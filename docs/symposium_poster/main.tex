\documentclass[final]{beamer}

\usepackage[scale=1.24]{beamerposter} % Use the beamerposter package for laying out the poster

\usetheme{confposter} % Use the confposter theme supplied with this template

\setbeamercolor{block title}{fg=black,bg=white} % Colors of the block titles
\setbeamercolor{block body}{fg=black,bg=white} % Colors of the body of blocks
\setbeamercolor{block alerted title}{fg=white,bg=dblue!70} % Colors of the highlighted block titles
\setbeamercolor{block alerted body}{fg=black,bg=dblue!10} % Colors of the body of highlighted blocks
% Many more colors are available for use in beamerthemeconfposter.sty

%-----------------------------------------------------------
% Define the column widths and overall poster size
% To set effective sepwid, onecolwid and twocolwid values, first choose how many columns you want and how much separation you want between columns
% In this template, the separation width chosen is 0.024 of the paper width and a 4-column layout
% onecolwid should therefore be (1-(# of columns+1)*sepwid)/# of columns e.g. (1-(4+1)*0.024)/4 = 0.22
% Set twocolwid to be (2*onecolwid)+sepwid = 0.464
% Set threecolwid to be (3*onecolwid)+2*sepwid = 0.708

\newlength{\sepwid}
\newlength{\onecolwid}
\newlength{\twocolwid}
\newlength{\threecolwid}
\setlength{\paperwidth}{48in} % A0 width: 46.8in
\setlength{\paperheight}{36in} % A0 height: 33.1in
\setlength{\sepwid}{0.024\paperwidth} % Separation width (white space) between columns
\setlength{\onecolwid}{0.22\paperwidth} % Width of one column
\setlength{\twocolwid}{0.464\paperwidth} % Width of two columns
\setlength{\threecolwid}{0.708\paperwidth} % Width of three columns
\setlength{\topmargin}{-0.5in} % Reduce the top margin size


\setbeamertemplate{headline}{
 \leavevmode
  \begin{columns}
  \begin{column}{.04\linewidth}
  \end{column}
     \begin{column}{.15\linewidth}
\hfill \includegraphics[width=1\linewidth]{logo.png} 
   \end{column}
   \begin{column}{.73\linewidth}
    \vskip1cm
    \centering
    \usebeamercolor{title in headline}{\color{jblue}\Huge{\textbf{\inserttitle}}\\[0.5ex]}
    \usebeamercolor{author in headline}{\color{fg}\Large{\insertauthor}\\[1ex]}
    \usebeamercolor{institute in headline}{\color{fg}\large{\insertinstitute}\\[1ex]}
    \vskip1cm
   \end{column}
   \begin{column}{.17\linewidth}
   \end{column}
   \vspace{1cm}
  \end{columns}
 \vspace{0.5in}
 \hspace{0.5in}\begin{beamercolorbox}[wd=47in,colsep=0.15cm]{cboxb}\end{beamercolorbox}
 \vspace{0.1in}
}

%-----------------------------------------------------------

\usepackage{graphicx}  % Required for including images

\usepackage{booktabs} % Top and bottom rules for tables

%----------------------------------------------------------------------------------------
%	TITLE SECTION 
%----------------------------------------------------------------------------------------

\title{Your Awesome Research: ML@B  Poster Template} % Poster title

\author{William Guss} % Author(s)

\institute{Machine Learning at Berkeley} % Institution(s)

%----------------------------------------------------------------------------------------

\begin{document}

\addtobeamertemplate{block end}{}{\vspace*{2ex}} % White space under blocks
\addtobeamertemplate{block alerted end}{}{\vspace*{2ex}} % White space under highlighted (alert) blocks

\setlength{\belowcaptionskip}{2ex} % White space under figures
\setlength\belowdisplayshortskip{2ex} % White space under equations

\begin{frame}[t] % The whole poster is enclosed in one beamer frame

\begin{columns}[t] % The whole poster consists of three major columns, the second of which is split into two columns twice - the [t] option aligns each column's content to the top

\begin{column}{\sepwid}\end{column} % Empty spacer column

\begin{column}{\onecolwid} % The first column

%----------------------------------------------------------------------------------------
%	OBJECTIVES
%----------------------------------------------------------------------------------------

\begin{alertblock}{Abstract}

In this project, we generalize ANNs to infinite dimensional Banach spaces by developing a
practical analog to the feedforward propagation algorithm. Using this new class of algorithms, GANN, 
we prove a new universal approximation theorem for bounded linear operators and show that representation
of weights by samples from a multivariate weight polynomial can drastically reduce the dimensionality of a
learning problem. Lastly, we give a practical implementation of the error back-propagation algorithm in this
space for the classification of continuous data.
\end{alertblock}

%----------------------------------------------------------------------------------------
%	INTRODUCTION
%----------------------------------------------------------------------------------------

\begin{block}{Introduction}

Although neural networks have proven an extremely effective mechanism of machine learning [1], theoretically they remain a black-box model.  In answer to this problem [2] examined the notion of infinite hidden nodes with a network proving that such a construction becomes a Gaussian kernel. Then, [2] described a model for affine neural networks with continuous hidden layers in alignment with [1]. These authors showed effectively the viability of a "continuous" neural network, but left many similar constructions unexplored. 

\cite{Smith:2012qr}.

\end{block}

%------------------------------------------------


 \begin{block}{Artificial Neural Networks}
   \textbf{Definition 1.} \emph{ 
    We say $\mathcal{N}: \mathbb{R}^n \to \mathbb{R}^m$ is a feed-forward neural network if for an input vector $\pmb{x}$,
    \begin{equation}
            \begin{aligned}
        \mathcal{N}:\ & \sigma_j^{(l+1)} &= g\left(\sum_{i \in Z^{(l)}}w_{ij}^{(l)}\sigma_i^{(l)} + \beta^{(l)}\right) \\ & \sigma_j^{(1)} &= g\left(\sum_{i \in Z^{(0)}}w_{ij}^{(0)}x_i + \beta^{(0)} \right),
        \end{aligned}
    \end{equation}
    where $1\leq l \leq L-1$. Furthermore we say $\{\mathcal{N}\}$ is the set of all neural networks.}

\end{block}
%----------------------------------------------------------------------------------------

\end{column} % End of the first column

\begin{column}{\sepwid}\end{column} % Empty spacer column

\begin{column}{\twocolwid} % Begin a column which is two columns wide (column 2)

\begin{columns}[t,totalwidth=\twocolwid] % Split up the two columns wide column

\begin{column}{\onecolwid}\vspace{-.6in} % The first column within column 2 (column 2.1)

%----------------------------------------------------------------------------------------
%	MATERIALS
%----------------------------------------------------------------------------------------

\begin{block}{Functional Neural Networks}

  Suppose that we wish to map one functional space to another with a neural network. Consider the standard model of an ANN as the number of neural nodes for every layer becomes uncountable. The index for each node then becomes real-valued, along with the weight and input vectors.\\[1cm]

  \textbf{Definition 2.} \emph{We call $\mathcal{F}: L^1(X) \to L^1(Y)$ a functional neural network if,
  \begin{equation}
            \begin{alignedat}{2}
          \mathcal{F}:\ &\sigma^{(l+1)}(j) & &=  g\left(\int_{R^{(l)}} \sigma^{(l)}(i) w^{(l)}(i,j)\ di \right)  \\
          &\sigma^{(0)}(j) & &= \xi(j). 
          \end{alignedat}
  \end{equation}
  Furthermore let $\{\mathcal{F}\}$ denote the set of all functional neural networks.}

\end{block}

%----------------------------------------------------------------------------------------

\end{column} % End of column 2.1

\begin{column}{\onecolwid}\vspace{-.6in} % The second column within column 2 (column 2.2)

%----------------------------------------------------------------------------------------
%	METHODS
%----------------------------------------------------------------------------------------

\begin{block}{Methods}

Lorem ipsum dolor \textbf{sit amet}, consectetur adipiscing elit. Sed laoreet accumsan mattis. Integer sapien tellus, auctor ac blandit eget, sollicitudin vitae lorem. Praesent dictum tempor pulvinar. Suspendisse potenti. Sed tincidunt varius ipsum, et porta nulla suscipit et. Etiam congue bibendum felis, ac dictum augue cursus a. \textbf{Donec} magna eros, iaculis sit amet placerat quis, laoreet id est. In ut orci purus, interdum ornare nibh. Pellentesque pulvinar, nibh ac malesuada accumsan, urna nunc convallis tortor, ac vehicula nulla tellus eget nulla. Nullam lectus tortor, \textit{consequat tempor hendrerit} quis, vestibulum in diam. Maecenas sed diam augue.

\end{block}

%----------------------------------------------------------------------------------------

\end{column} % End of column 2.2

\end{columns} % End of the split of column 2 - any content after this will now take up 2 columns width

%----------------------------------------------------------------------------------------
%	IMPORTANT RESULT
%----------------------------------------------------------------------------------------

\begin{alertblock}{Important Result}
\textbf{Theorem.} \emph{Given a functional neural network $\mathcal{F}$ then some layer $l \in \mathcal{F}$, the let $K:C(R^{(l)})\to C(R^{(l)})$ be a bounded linear operator. If we denote the operation of layer $l$ on layer $l-1$ as $\sigma^{(l+1)} = g\left(\Sigma_{l+1}\sigma^{(l)}\right)$, then for every $\epsilon >0$, there exists a weight polynomial $w^{(l)}(i,j)$ such that the supremum norm over }$R^{(l)}$ \begin{equation}\left\|K\sigma^{(l)} -\Sigma_{l+1}\sigma^{(l)}\right\|_{\infty} < \epsilon\end{equation}

\end{alertblock} 

%----------------------------------------------------------------------------------------

\begin{columns}[t,totalwidth=\twocolwid] % Split up the two columns wide column again

\begin{column}{\onecolwid} % The first column within column 2 (column 2.1)

%----------------------------------------------------------------------------------------
%	MATHEMATICAL SECTION
%----------------------------------------------------------------------------------------

\begin{block}{Mathematical Section}

Nam quis odio enim, in molestie libero. Vivamus cursus mi at nulla elementum sollicitudin. Nam quis odio enim, in molestie libero. Vivamus cursus mi at nulla elementum sollicitudin.
  
\begin{equation}
E = mc^{2}
\label{eqn:Einstein}
\end{equation}

Nam quis odio enim, in molestie libero. Vivamus cursus mi at nulla elementum sollicitudin. Nam quis odio enim, in molestie libero. Vivamus cursus mi at nulla elementum sollicitudin.

\begin{equation}
\cos^3 \theta =\frac{1}{4}\cos\theta+\frac{3}{4}\cos 3\theta
\label{eq:refname}
\end{equation}



\end{block}

%----------------------------------------------------------------------------------------

\end{column} % End of column 2.1

\begin{column}{\onecolwid} % The second column within column 2 (column 2.2)

%----------------------------------------------------------------------------------------
%	RESULTS
%----------------------------------------------------------------------------------------

\begin{block}{Results}

\begin{figure}
\includegraphics[width=0.8\linewidth]{placeholder.jpg}
\caption{Figure caption}
\end{figure}

Nunc tempus venenatis facilisis. Curabitur suscipit consequat eros non porttitor. Sed a massa dolor, id ornare enim:

\end{block}

%----------------------------------------------------------------------------------------

\end{column} % End of column 2.2

\end{columns} % End of the split of column 2

\end{column}

\begin{column}{\sepwid}\end{column} % Empty spacer column


\begin{column}{\onecolwid} % The third column

%----------------------------------------------------------------------------------------
%	CONCLUSION
%----------------------------------------------------------------------------------------

\begin{block}{Conclusion}

Nunc tempus venenatis facilisis. \textbf{Curabitur suscipit} consequat eros non porttitor. Sed a massa dolor, id ornare enim. Fusce quis massa dictum tortor \textbf{tincidunt mattis}. Donec quam est, lobortis quis pretium at, laoreet scelerisque lacus. Nam quis odio enim, in molestie libero. Vivamus cursus mi at \textit{nulla elementum sollicitudin}.

\end{block}

%----------------------------------------------------------------------------------------
%	ADDITIONAL INFORMATION
%----------------------------------------------------------------------------------------

\begin{block}{Additional Information}

Maecenas ultricies feugiat velit non mattis. Fusce tempus arcu id ligula varius dictum. 
\begin{itemize}
\item Curabitur pellentesque dignissim
\item Eu facilisis est tempus quis
\item Duis porta consequat lorem
\end{itemize}

\end{block}

%----------------------------------------------------------------------------------------
%	REFERENCES
%----------------------------------------------------------------------------------------

\begin{block}{References}

\nocite{*} % Insert publications even if they are not cited in the poster
\small{\bibliographystyle{unsrt}
\bibliography{sample}\vspace{0.75in}}

\end{block}

%----------------------------------------------------------------------------------------
%	ACKNOWLEDGEMENTS
%----------------------------------------------------------------------------------------

\setbeamercolor{block title}{fg=red,bg=white} % Change the block title color

\begin{block}{Acknowledgements}

\small{\rmfamily{Nam mollis tristique neque eu luctus. Suspendisse rutrum congue nisi sed convallis. Aenean id neque dolor. Pellentesque habitant morbi tristique senectus et netus et malesuada fames ac turpis egestas.}} \\

\end{block}

%----------------------------------------------------------------------------------------
%	CONTACT INFORMATION
%----------------------------------------------------------------------------------------

\setbeamercolor{block alerted title}{fg=black,bg=norange} % Change the alert block title colors
\setbeamercolor{block alerted body}{fg=black,bg=white} % Change the alert block body colors

\begin{alertblock}{Contact Information}

\begin{itemize}
\item Web: \href{http://www.university.edu/smithlab}{http://www.university.edu/smithlab}
\item Email: \href{mailto:john@smith.com}{john@smith.com}
\item Phone: +1 (000) 111 1111
\end{itemize}

\end{alertblock}



%----------------------------------------------------------------------------------------

\end{column} % Endz of the third column

\end{columns} % End of all the columns in the poster

\end{frame} % End of the enclosing frame

\end{document}
